% This is LLNCS.DEM the demonstration file of
% the LaTeX macro package from Springer-Verlag
% for Lecture Notes in Computer Science,
% version 2.4 for LaTeX2e as of 16. April 2010
%
\documentclass{llncs}
%
\usepackage{makeidx}  % allows for indexgeneration
\usepackage{color} %vrivas, 15-Jan-2016
%c
\begin{document}
%
\frontmatter          % for the preliminaries
%
\pagestyle{headings}  % switches on printing of running heads
\addtocmark{Web browser-based forecasting} % additional mark in the TOC
%
%\chapter*{Preface}
%%
%This textbook is intended for use by students of physics, physical
%chemistry, and theoretical chemistry. The reader is presumed to have a
%basic knowledge of atomic and quantum physics at the level provided, for
%example, by the first few chapters in our book {\it The Physics of Atoms
%and Quanta}. The student of physics will find here material which should
%be included in the basic education of every physicist. This book should
%furthermore allow students to acquire an appreciation of the breadth and
%variety within the field of molecular physics and its future as a
%fascinating area of research.
%
%For the student of chemistry, the concepts introduced in this book will
%provide a theoretical framework for that entire field of study. With the
%help of these concepts, it is at least in principle possible to reduce
%the enormous body of empirical chemical knowledge to a few basic
%principles: those of quantum mechanics. In addition, modern physical
%methods whose fundamentals are introduced here are becoming increasingly
%important in chemistry and now represent indispensable tools for the
%chemist. As examples, we might mention the structural analysis of
%complex organic compounds, spectroscopic investigation of very rapid
%reaction processes or, as a practical application, the remote detection
%of pollutants in the air.

%\vspace{1cm}
%\begin{flushright}\noindent
%April 1995\hfill Walter Olthoff\\
%Program Chair\\
%ECOOP'95
%\end{flushright}
%%
%\chapter*{Organization}
%ECOOP'95 is organized by the department of Computer Science, Univeristy
%of \AA rhus and AITO (association Internationa pour les Technologie
%Object) in cooperation with ACM/SIGPLAN.
%%
%\section*{Executive Commitee}
%\begin{tabular}{@{}p{5cm}@{}p{7.2cm}@{}}
%Conference Chair:&Ole Lehrmann Madsen (\AA rhus University, DK)\\
%Program Chair:   &Walter Olthoff (DFKI GmbH, Germany)\\
%Organizing Chair:&J\o rgen Lindskov Knudsen (\AA rhus University, DK)\\
%Tutorials:&Birger M\o ller-Pedersen\hfil\break
%(Norwegian Computing Center, Norway)\\
%Workshops:&Eric Jul (University of Kopenhagen, Denmark)\\
%Panels:&Boris Magnusson (Lund University, Sweden)\\
%Exhibition:&Elmer Sandvad (\AA rhus University, DK)\\
%Demonstrations:&Kurt N\o rdmark (\AA rhus University, DK)
%\end{tabular}
%%
%\section*{Program Commitee}
%\begin{tabular}{@{}p{5cm}@{}p{7.2cm}@{}}
%Conference Chair:&Ole Lehrmann Madsen (\AA rhus University, DK)\\
%Program Chair:   &Walter Olthoff (DFKI GmbH, Germany)\\
%Organizing Chair:&J\o rgen Lindskov Knudsen (\AA rhus University, DK)\\
%Tutorials:&Birger M\o ller-Pedersen\hfil\break
%(Norwegian Computing Center, Norway)\\
%Workshops:&Eric Jul (University of Kopenhagen, Denmark)\\
%Panels:&Boris Magnusson (Lund University, Sweden)\\
%Exhibition:&Elmer Sandvad (\AA rhus University, DK)\\
%Demonstrations:&Kurt N\o rdmark (\AA rhus University, DK)
%\end{tabular}
%%
%\begin{multicols}{3}[\section*{Referees}]
%V.~Andreev\\
%B\"arwolff\\
%E.~Barrelet\\
%H.P.~Beck\\
%G.~Bernardi\\
%E.~Binder\\
%P.C.~Bosetti\\
%Braunschweig\\
%F.W.~B\"usser\\
%T.~Carli\\
%A.B.~Clegg\\
%G.~Cozzika\\
%S.~Dagoret\\
%Del~Buono\\
%P.~Dingus\\
%H.~Duhm\\
%J.~Ebert\\
%S.~Eichenberger\\
%R.J.~Ellison\\
%Feltesse\\
%W.~Flauger\\
%A.~Fomenko\\
%G.~Franke\\
%J.~Garvey\\
%M.~Gennis\\
%L.~Goerlich\\
%P.~Goritchev\\
%H.~Greif\\
%E.M.~Hanlon\\
%R.~Haydar\\
%R.C.W.~Henderso\\
%P.~Hill\\
%H.~Hufnagel\\
%A.~Jacholkowska\\
%Johannsen\\
%S.~Kasarian\\
%I.R.~Kenyon\\
%C.~Kleinwort\\
%T.~K\"ohler\\
%S.D.~Kolya\\
%P.~Kostka\\
%U.~Kr\"uger\\
%J.~Kurzh\"ofer\\
%M.P.J.~Landon\\
%A.~Lebedev\\
%Ch.~Ley\\
%F.~Linsel\\
%H.~Lohmand\\
%Martin\\
%S.~Masson\\
%K.~Meier\\
%C.A.~Meyer\\
%S.~Mikocki\\
%J.V.~Morris\\
%B.~Naroska\\
%Nguyen\\
%U.~Obrock\\
%G.D.~Patel\\
%Ch.~Pichler\\
%S.~Prell\\
%F.~Raupach\\
%V.~Riech\\
%P.~Robmann\\
%N.~Sahlmann\\
%P.~Schleper\\
%Sch\"oning\\
%B.~Schwab\\
%A.~Semenov\\
%G.~Siegmon\\
%J.R.~Smith\\
%M.~Steenbock\\
%U.~Straumann\\
%C.~Thiebaux\\
%P.~Van~Esch\\
%from Yerevan Ph\\
%L.R.~West\\
%G.-G.~Winter\\
%T.P.~Yiou\\
%M.~Zimmer\end{multicols}
%%
%\section*{Sponsoring Institutions}
%%
%Bernauer-Budiman Inc., Reading, Mass.\\
%The Hofmann-International Company, San Louis Obispo, Cal.\\
%Kramer Industries, Heidelberg, Germany
%
%%
%\tableofcontents
%
\mainmatter              % start of the contributions
%
\title{Web browser-based forecasting of economic time-series}
%
\titlerunning{Web browser-based forecasting}  % abbreviated title (for running head)
%                                     also used for the TOC unless
%                                     \toctitle is used
%
\author{
V. M. Rivas\inst{1} 
\and Other?\inst{2}
}
%
\authorrunning{V. M. Rivas et al.} % abbreviated author list (for running head)
%
%%%% list of authors for the TOC (use if author list has to be modified)
\tocauthor{V M. Rivas}
%
\institute{Depto. de Informatica, Univ. de Jaen, SPAIN,\\
\email{vrivas@ujaen.es},\\ 
\texttt{http://vrivas.es}
\and
Depto. de Arquitectura y Tecnolog\'{\i}as de las Computadoras\\
Univ. de Granada, SPAIN
}

\maketitle              % typeset the title of the contribution
\textbf{Selling point: Economic time-series (currency exchange) can be forecasted using web browser while user are reading the content of a web-page. Lot of clients running simultaneously can lead to find very good solutions.}
\begin{abstract}
{\color{red}
The abstract should summarize the contents of the paper
using at least 70 and at most 150 words. It will be set in 9-point
font size and be inset 1.0 cm from the right and left margins.
There will be two blank lines before and after the Abstract. \dots
}
\keywords{Time-series forecasting, evolutionary computation, radial basis function neural networks, Web-based programming, volunteer computation}
\end{abstract}
%
\section{Introduction}
\section{State of the art}
\section{Algorithm}
\section{Experiments}
\section{Results}
\section{Conclusions}
%
% ---- Bibliography ----
%
\begin{thebibliography}{}
%
%\bibitem[1980]{2clar:eke}
%Clarke, F., Ekeland, I.:
%Nonlinear oscillations and
%boundary-value problems for Hamiltonian systems.
%Arch. Rat. Mech. Anal. 78, 315--333 (1982)


\end{thebibliography}
%\clearpage
%\addtocmark[2]{Author Index} % additional numbered TOC entry
%\renewcommand{\indexname}{Author Index}
%\printindex
%\clearpage
%\addtocmark[2]{Subject Index} % additional numbered TOC entry
%\markboth{Subject Index}{Subject Index}
%\renewcommand{\indexname}{Subject Index}
%\input{subjidx.ind}
\end{document}
